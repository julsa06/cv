\documentclass{cv}

\usepackage{datetime}
\newdateformat{monthyear}{\monthname[\THEMONTH], \THEYEAR}

\usepackage{fontawesome}
\usepackage{fontspec}
\usepackage[russian,english]{babel}
\setmainfont[Ligatures=TeX]{Linux Libertine O}
\setmonofont[Ligatures=TeX]{Linux Libertine Mono O}
\setsansfont[Ligatures=TeX]{Liberation Sans}

\begin{document}

\cvheading{Юлия Абдрашитова}

\section{Контактная информация}

\entry{\faEnvelope}{\href{mailto:julsa06@gmail.com}{julsa06@gmail.com}}

\entry{\faPhone}{+7 (999) 535 3557}

\entry{\faMapMarker}{Санкт-Петербург, Россия}

\entry{\faGithubAlt}{\href{https://github.com/julsa06}{github.com/julsa06}}

\section{Опыт}

\begin{cvblock}{%
  \blocktitle{T-Systems RUS}%
             {г. Санкт-Петербург}%
             {Младший инженер-программист}%
             {Февраль--Май 2016}}
  Система для подачи и управления заявками на обучение для сотрудников 
  компании в рамках университетской практики.
  \begin{itemize}
      \item Проектная деятельность: Участие в проектировании системы.
            Разработка системы;
      \item Используемые технологии: Java, Hibernate, Spring, Thymeleaf, MySQL,
            JavaScript.
  \end{itemize}
\end{cvblock}

\vspace{2em}

\begin{cvblock}{%
  \blocktitle{Университет ИТМО}%
             {г. Санкт-Петербург}%
             {Ассистент}%
             {с 2017}}
  Курс по базовым алгоритмам и структурам данных на кафедре высшей математики.
  \begin{itemize}
      \item Составление программы курса;
      \item проведение лекций и практики по курсу.
  \end{itemize}
\end{cvblock}

\vspace{2em}

\begin{cvblock}{%
  \blocktitle{Университет Иннополис}%
             {г. Иннополис}%
             {Преподаватель}%
             {с 2016}}
  \begin{itemize}
    \item Преподаватель школы олимпиадной подготовки по информатике:
          \begin{itemize}
              \item подготовка и разработка учебных материалов (учебный план,
                    контесты),
              \item проведение лекций и практических занятий,
              \item составление и проверка вступительной работы;
         \end{itemize}
    \item Член жюри открытой олимпиады университета Иннополис по информатике:
          \begin{itemize}
              \item составление задач,
              \item помощь в организации и проведении.
         \end{itemize}
  \end{itemize}
\end{cvblock}

\vspace{2em}

\begin{cvblock}{%
  \blocktitle{Физико-Математический лицей \textnumero~366}%
             {г. Санкт-Петербург}%
             {Педагог доп. образования}%
             {с 2011}}

  \begin{itemize}
    \item Организация и проведение школьных и муниципальных этапов 
          всероссийской олимпиады школьников по программированию;
    \item Проведение кружков по олимпиадному программированию. Обучение 
          школьников основам ЯП (java, pascal) и базовым алгоритмам и
          структурам данных.
  \end{itemize}
\end{cvblock}

\vspace{2em}

\begin{cvblock}{Прочее}
  \begin{itemize}
      \item Волонтер на олимпиадах: \begin{itemize}
          \item ACM ICPC World finals 2013, 
          \item полуфинал (NEERC) и четвертьфинал ACM ICPC с 2012,
          \item чемпионаты и кубки СПбГУ по программированию,
          \item школьные олимпиады: ВКОШП, ИОИП, региональный этап Всеросийской
                олимпиады по информатике, турнир Архимеда, Матпраздник и др;
      \end{itemize}
      \item Преподаватель (введение в олимпиадные программирование на языке
            python) и вожатый в летней МультиМатической Школе 2016~г.;
      \item Составление олимпиады по информатике для онлайн-школы Фоксфорд.
  \end{itemize}
\end{cvblock}


\section{Образование}

\begin{cvblock}{%
  \blocktitle
    {Университет ИТМО}
    {г. Санкт-Петербург}
    {}
    {с 2017}}
  Кафедра высшей математики. Математическое моделирование.
  \vspace{1em}

  \textit{Магистр прикладной математики и информатики.}
\end{cvblock}

\begin{cvblock}{%
  \blocktitle
    {Европейский университет "Бизнес Треугольник"}
    {г. Санкт-Петербург}
    {}
    {2016--2017}}
  Педагогическое образование: учитель информатики и ИКТ.
  \vspace{1em}

  \textit{Учитель информатики и ИКТ.}
\end{cvblock}

\begin{cvblock}{%
  \blocktitle
    {Университет ИТМО}
    {г. Санкт-Петербург}
    {}
    {2012--2016}}
  Кафедра высшей математики. Математическое моделирование.
  \vspace{1em}

  \textit{Бакалавр прикладной математики и информатики.}
\end{cvblock}


\vfill
\begin{center}
  \monthyear\today
\end{center}

\end{document}
